%%% DOCUMENTCLASS 
%%%-------------------------------------------------------------------------------
\documentclass[
a4paper,     % Stock and paper size.
12pt,        % Type size.
article,
%oneside, 
onecolumn,   % Only one column of text on a page.
% openright, % Each chapter will start on a recto page.
% openleft,  % Each chapter will start on a verso page.
openany,     % A chapter may start on either a recto or verso page.
]{memoir}


%%% PACKAGES 
%%%------------------------------------------------------------------------------
% If utf8 encoding
\usepackage[utf8]{inputenc}

% If not utf8 encoding, then this is probably the way to go
\usepackage[T1]{fontenc}
\usepackage{lmodern}

% English please
\usepackage[english]{babel}

% Less badboxes
\usepackage[final]{microtype}

% Math
\usepackage{amsmath,amssymb,mathtools}

% http://ctan.org/pkg/pifont
\usepackage{pifont}
\newcommand{\cmark}{\ding{51}}
\newcommand{\xmark}{\ding{55}}

% Include figures
\usepackage{graphicx}
\usepackage{makeidx}
\usepackage{import}
\usepackage[]{algorithm2e}
\usepackage{algpseudocode}
\RestyleAlgo{boxruled}

%%% PAGE LAYOUT 
%%%-----------------------------------------------------------------------------
% Left and right margin
\setlrmarginsandblock{0.15\paperwidth}{*}{1}
% Upper and lower margin
\setulmarginsandblock{0.2\paperwidth}{*}{1}

\setpnumwidth{3em}
\setrmarg{4em}
\setlength{\cftpartnumwidth}{3em}
\checkandfixthelayout

\newlength\forceindent
\setlength{\forceindent}{\parindent}
\setlength{\parindent}{0cm}
\renewcommand{\indent}{\hspace*{\forceindent}}
\setlength{\parskip}{1em}

%%% SECTIONAL DIVISIONS
%%%------------------------------------------------------------------------------
% Subsections (and higher) are numbered
\maxsecnumdepth{paragraph}
\setsecnumdepth{paragraph}

\usepackage{titlesec}
\usepackage{etoolbox}

\titlespacing\section{0pt}{12pt plus 4pt minus 2pt}{0pt plus 2pt minus 2pt}
\titlespacing\subsection{0pt}{12pt plus 2pt minus 1pt}{0pt plus 1pt minus 1pt}
\titlespacing\subsubsection{0pt}{12pt plus 2pt minus 1pt}{0pt plus 1pt minus 1pt}

\counterwithout{section}{chapter}
\titleformat*{\section}{\Large\bfseries}
%\patchcmd{\thebibliography}{\chapter*}{\section}{}{}

\setsecheadstyle{\normalfont\large\bfseries}
\setsubsecheadstyle{\normalfont\normalsize\bfseries}
\setparaheadstyle{\normalfont\normalsize\bfseries}
\setparaindent{0pt}\setafterparaskip{0pt}

%%% FLOATS AND CAPTIONS
%%%------------------------------------------------------------------------------
\newcommand{\fig}[1]{figure~{\ref{#1}}\xspace}
\newcommand{\tab}[1]{table~{\ref{#1}}\xspace}
\newcommand{\lista}[1]{listing~{\ref{#1}}\xspace}
\newcommand{\parte}[1]{part~{\ref{#1}}\xspace}
\newcommand{\chap}[1]{chapter~{\ref{#1}}\xspace}
\newcommand{\sect}[1]{section~{\ref{#1}}\xspace}

% A space between caption name and text
\captiondelim{\space}
% Font of the caption name
\captionnamefont{\small\bfseries}
% Font of the caption text
\captiontitlefont{\small\normalfont}

% Change the width of the caption
\changecaptionwidth        
\captionwidth{1\textwidth} 

\usepackage{tabularx}
%\usepackage{float}

%%% ABSTRACT
%%%------------------------------------------------------------------------------
% Font of abstract title
\renewcommand{\abstractnamefont}{\normalfont\small\bfseries}
% Width of abstract 
\setlength{\absleftindent}{0.1\textwidth}
\setlength{\absrightindent}{\absleftindent}

%%% HEADER AND FOOTER 
%%%------------------------------------------------------------------------------
% Make standard pagestyle
\makepagestyle{memoirStylePages}
\makerunningwidth{memoirStylePages}{\textwidth}

\makeheadrule{memoirStylePages}{\textwidth}{\normalrulethickness}
\makefootrule{memoirStylePages}{\textwidth}{\normalrulethickness}{0pt}

\makeevenfoot{memoirStylePages}{}{\bfseries\thepage}{}
\makeoddfoot{memoirStylePages}{}{\bfseries\thepage}{}
\makeevenhead{memoirStylePages}{}{\textit{Dwarf 3: Cloud scheme}}{}
\makeoddhead{memoirStylePages}{}{\textit{Dwarf 3: Cloud scheme}}{}

\makeatletter
\makepsmarks{memoirStylePages}{
\createmark{chapter}{both}{shownumber}{\@chapapp\ }{ \quad }
\createmark{section}{right}{shownumber}{}{ \quad }
\createplainmark{toc}{both}{\contentsname}
\createplainmark{lof}{both}{\listfigurename}
\createplainmark{lot}{both}{\listtablename}
\createplainmark{bib}{both}{\bibname}
\createplainmark{index}{both}{\indexname}
\createplainmark{glossary}{both}{\glossaryname}
}
\makeatother

% Choosing pagestyle and chapter pagestyle
\pagestyle{memoirStylePages}
\aliaspagestyle{chapter}{chap}


%%% NEW COMMANDS
%%%-----------------------------------------------------------------------------

% Nektar++ version
\usepackage{xspace}
\ifdefined\HCode
\newcommand{\version}{0.1.0\unskip}
\else
\newcommand{\version}{0.1.0
\unskip}
\fi
\addto\captionsenglish{\renewcommand{\chaptername}{Chapter}}

% Partial
\newcommand{\p}{\partial}



%%% CODE SNIPPETS, COMMANDS, ETC
%%%-----------------------------------------------------------------------------
\usepackage{xcolor}
\usepackage{tikz}
\definecolor{gray}{rgb}{0.4,0.4,0.4}
\definecolor{lightgray}{rgb}{0.9,0.9,0.9}
\definecolor{darkblue}{rgb}{0.0,0.0,0.6}
\definecolor{cyan}{rgb}{0.0,0.6,0.6}
\definecolor{maroon}{rgb}{0.5,0.0,0.0}
\definecolor{darkgreen}{rgb}{0.0,0.5,0.0}

% Display code / shell commands
\usepackage{listings}
\usepackage{lstautogobble}


% Bash input style
\lstdefinestyle{BashStyle}
{
  language=bash,
  basicstyle=\footnotesize\ttfamily,
%numbers=left,
%numberstyle=\tiny,
%numbersep=3pt,
  frame=single,
  columns=fullflexible,
  backgroundcolor=\color{yellow!10},
  linewidth=\linewidth,
  xleftmargin=0.05\linewidth,
  keepspaces=true,
  framesep=5pt,
  rulecolor=\color{black!30},
  aboveskip=10pt,
  autogobble=true
}


% XML input style definition
\lstdefinelanguage{XML}
{
  basicstyle=\ttfamily\footnotesize,
  morestring=[b]",
  moredelim=[s][\bfseries\color{maroon}]{<}{\ },
  moredelim=[s][\bfseries\color{maroon}]{</}{>},
  moredelim=[l][\bfseries\color{maroon}]{/>},
  moredelim=[l][\bfseries\color{maroon}]{>},
  morecomment=[s]{<?}{?>},
  morecomment=[s]{<!--}{-->},
  commentstyle=\color{gray},
  stringstyle=\color{orange},
  identifierstyle=\color{darkblue},
  showstringspaces=false
}


% XML input style
\lstdefinestyle{XMLStyle}
{
  language=make,
  basicstyle=\ttfamily\footnotesize,
  numbers=left,
  numberstyle=\tiny,
  numbersep=3pt,
  frame=,
  columns=fullflexible,
  backgroundcolor=\color{black!05},
  linewidth=\linewidth,
  xleftmargin=0.05\linewidth,
  keepspaces=true
}
\lstset{
    escapeinside={(*}{*)},
}


\lstdefinestyle{CStyle}{
  belowcaptionskip=1\baselineskip,
  breaklines=true,
  frame=single,
  escapeinside={\%*}{*)},
  xleftmargin=\parindent,
  language=C,
  captionpos=b,
  keepspaces=true,
  backgroundcolor=\color{black!05},
  showstringspaces=false,
  numbers=left,
  numbersep=5pt,
  numberstyle=\tiny\color{black},
  basicstyle=\scriptsize\ttfamily,
  keywordstyle=\bfseries\color{green!40!black},
  commentstyle=\itshape\color{purple!40!black},
  identifierstyle=\color{blue},
  stringstyle=\color{orange},
  tabsize=4
}

\lstdefinestyle{CStyleNoLine}{
  belowcaptionskip=1\baselineskip,
  breaklines=true,
  frame=single,
  escapeinside={\%*}{*)},
  xleftmargin=\parindent,
  language=C,
  captionpos=b,
  keepspaces=true,
  backgroundcolor=\color{black!05},
  showstringspaces=false,
  basicstyle=\scriptsize\ttfamily,
  keywordstyle=\bfseries\color{green!40!black},
  commentstyle=\itshape\color{purple!40!black},
  identifierstyle=\color{blue},
  stringstyle=\color{orange},
  tabsize=4
}


\lstdefinestyle{FStyle}{
  belowcaptionskip=1\baselineskip,
  breaklines=true,
  frame=single,
  escapeinside={\%*}{*)},
  xleftmargin=\parindent,
  language=[90]Fortran,
  captionpos=b,
  keepspaces=true,
  backgroundcolor=\color{red!05},
  showstringspaces=false,
  numbers=left,
  numbersep=5pt,
  numberstyle=\tiny\color{black},
  basicstyle=\scriptsize\ttfamily,
  keywordstyle=\bfseries\color{red!40!black},
  commentstyle=\itshape\color{green!40!black},
  identifierstyle=\color{blue},
  stringstyle=\color{orange},
  tabsize=4
}

\lstdefinestyle{FStyleNoLine}{
  belowcaptionskip=1\baselineskip,
  breaklines=true,
  frame=single,
  escapeinside={\%*}{*)},
  xleftmargin=\parindent,
  language=[90]Fortran,
  captionpos=b,
  keepspaces=true,
  backgroundcolor=\color{red!05},
  showstringspaces=false,
  basicstyle=\scriptsize\ttfamily,
  keywordstyle=\bfseries\color{red!40!black},
  commentstyle=\itshape\color{green!40!black},
  identifierstyle=\color{blue},
  stringstyle=\color{orange},
  tabsize=4
}

% Inline commands for C++ and Fortran
\ifdefined\HCode
\newcommand{\inltc}[1]{\texttt{#1}}
\newcommand{\inltf}[1]{\texttt{#1}}
\else
\newcommand{\inltc}[1]{\tikz[anchor=base,baseline]\node[inner sep=2pt,
outer sep=0,fill=black!05,text=black]{\small\texttt{#1}};}
\newcommand{\inltf}[1]{\tikz[anchor=base,baseline]\node[inner sep=2pt,
outer sep=0,fill=red!05,text=black]{\small\texttt{#1}};}
\fi


% Inline commands for general words
\ifdefined\HCode
\newcommand{\inlsh}[1]{\texttt{#1}}
\else
\newcommand{\inlsh}[1]{\tikz[anchor=base,baseline]\node[inner sep=2pt,
outer sep=0,draw=yellow!10,fill=yellow!10]{\texttt{#1}};}
\fi

% double-dash
\newcommand \ddash{-\hspace{0.07em}-}

% Atlas
\newcommand{\Atlas}{{\em Atlas}\xspace}



% Highlight box
\usepackage{environ}
\usepackage[tikz]{bclogo}
\usetikzlibrary{calc}

% Only use fancy boxes for PDF
\ifdefined\HCode
\NewEnviron{notebox}{\textbf{Note:} \BODY}
\NewEnviron{warningbox}{\textbf{Warning:} \BODY}
\NewEnviron{tipbox}{\textbf{Tip:} \BODY}
\NewEnviron{custombox}[3]{\textbf{#1} \BODY}
\else
\NewEnviron{notebox}
  {\par\medskip\noindent
  \begin{tikzpicture}
    \node[inner sep=5pt,fill=black!10,draw=black!30] (box)
    {\parbox[t]{.99\linewidth}{%
      \begin{minipage}{.1\linewidth}
      \centering\tikz[scale=1]\node[scale=1.5]{\bcinfo};
      \end{minipage}%
      \begin{minipage}{.85\linewidth}
      \textbf{Note}\par\smallskip
      \BODY
      \end{minipage}\hfill}%
    };
   \end{tikzpicture}\par\medskip%
}
\NewEnviron{warningbox}
  {\par\medskip\noindent
  \begin{tikzpicture}
    \node[inner sep=5pt,fill=red!10,draw=black!30] (box)
    {\parbox[t]{.99\linewidth}{%
      \begin{minipage}{.1\linewidth}
      \centering\tikz[scale=1]\node[scale=1.5]{\bcdanger};
      \end{minipage}%
      \begin{minipage}{.85\linewidth}
      \textbf{Warning}\par\smallskip
      \BODY
      \end{minipage}\hfill}%
    };
   \end{tikzpicture}\par\medskip%
}
\NewEnviron{tipbox}
  {\par\medskip\noindent
  \begin{tikzpicture}
    \node[inner sep=5pt,fill=green!10,draw=black!30] (box)
    {\parbox[t]{.99\linewidth}{%
      \begin{minipage}{.1\linewidth}
      \centering\tikz[scale=1]\node[scale=1.5]{\bclampe};
      \end{minipage}%
      \begin{minipage}{.85\linewidth}
      \textbf{Tip}\par\smallskip
      \BODY
      \end{minipage}\hfill}%
    };
   \end{tikzpicture}\par\medskip%
}
\NewEnviron{custombox}[3]
  {\par\medskip\noindent
  \begin{tikzpicture}
    \node[inner sep=5pt,fill=#3!10,draw=black!30] (box)
    {\parbox[t]{.99\linewidth}{%
      \begin{minipage}{.1\linewidth}
      \centering\tikz[scale=1]\node[scale=1.5]{#2};
      \end{minipage}%
      \begin{minipage}{.85\linewidth}
      \textbf{#1}\par\smallskip
      \BODY
      \end{minipage}\hfill}%
    };
   \end{tikzpicture}\par\medskip%
}
\fi



%%% TABLE OF CONTENTS AND INDEX
%%%-----------------------------------------------------------------------------


\makeindex


%%% INTERNAL HYPERLINKS
%%%-----------------------------------------------------------------------------
% Internal hyperlinks
\usepackage[linktoc=all,hyperfootnotes=false]{hyperref}
\hypersetup{
colorlinks,
citecolor=darkblue,
filecolor=darkblue,
linkcolor=darkblue,
urlcolor=darkblue,
% No borders around internal hyperlinks
pdfborder={0 0 0},
% Author
pdfauthor={I am the Author}
}
\usepackage{memhfixc}


%%% PRETTY TITLE PAGE FOR PDF DOC
%%%-----------------------------------------------------------------------------
\ifdefined\HCode
\else
\makeatletter
\newlength\drop
\newcommand{\br}{\hfill\break}
\newcommand*{\titlepage}{
    \thispagestyle{empty}
    % Gentle Madness
    \begingroup
    \drop = 0.1\textheight
    \vspace*{\baselineskip}
    \vfill
    \hbox{
      \hspace*{0.1\textwidth}
      \rule{1pt}{\dimexpr\textheight-28pt\relax}
      \hspace*{0.05\textwidth}
      \parbox[b]{0.85\textwidth}{
        \vbox{
          \vspace{\drop}
          {\Huge\bfseries\raggedright\@title\par}
          \vskip1.10\baselineskip
          {\Large\bfseries Version \version\par}
          \vskip4\baselineskip
          {\huge\bfseries \textcolor{darkgreen}{Documentation}\par}
          \vskip1.0\baselineskip
          {\large\bfseries\@date\par}
          \vspace{0.2\textheight}
          {\small\noindent\@author}\\[\baselineskip]
        }% end of vbox
      }% end of parbox
    }% end of hbox
    \vfill
    \null
\endgroup}
\makeatother
\fi



%%% THE DOCUMENT
%%%-------------------------------------------------------------------------------
\author{ECMWF, Shinfield Park, Reading, UK\newline}
\title{Dwarf 3: Cloud Scheme}
\date{\today}


\begin{document}

\frontmatter

% Render pretty title page if not building HTML
\ifdefined\HCode
\maketitle
\begin{center}
    \huge{Version \version}
\end{center}
\else
\titlepage
\fi

\clearpage

\ifx\HCode\undefined
\tableofcontents*
\fi

\clearpage

\mainmatter


\section{Scope}
The cloud and precipitation microphysics is an essential building block 
of any weather and climate prediction model as it is necessary to represent 
the effects of small-scale sub-grid physical processes, such as cloud 
and precipitation microphysics, through a parameterization of the grid-scale 
prognostic variables in the model.
The cloud microphysics scheme is a computationally expensive routine and alternative 
ways that can accelerate its computation will be extremely beneficial 
for the computational performance of weather and climate prediction models.



\section{Objectives}
The main objective of this dwarf is to assess the scalability limits 
of the cloud scheme on different hardware such as hosts (e.g. CPUs) 
and devices (e.g. GPUs) and to test a hybrid combination of the two.

In addition, some new implementations of the cloud scheme are envisioned.
These are believed to perform better on device-type hardware, thereby 
providing a more energy-efficient and computationally faster solution 
for the cloud scheme.

More specfically, we aim to investigate the following points:
%
\begin{itemize}
\item multi-core (e.g. Broadwell) and many-core hosts (e.g. KNL) using 
multi-threading through OpenMP;
\item multi-device (e.g. multiple GPU nodes) implementing multi-threading 
using OpenACC, CUDA, etc;
\item explore different data-alignment / data-structure strategies 
to enhance compiler perfomance;
\item relax level-dependency by using data from previous time step 
that might provide finer grain parallelism (that would be beneficial 
for GPUs and KNL with OpenMP 4.5);
\end{itemize}
%
Each of the points outlined above will have a specific prototype 
implementation to permit a better understanding of the results, 
thus ultimately allowing the identification of the solution that 
guarantess the best compromise in terms of energy requirements 
/ time-to-solution.




\section{Definition of the Dwarf}
Weather and climate prediction models need to represent the effects of 
sub-grid scale physical processes. Radiation, turbulent 
mixing, convection, cloud and precipitation microphysics are 
examples of physical processes that are parametrized as a function 
of the grid-scale prognostic variables in models. This dwarf 
is the parametrizaton scheme for cloud and precipitation processes 
in the IFS, described by prognostic equations for cloud liquid water, 
cloud ice, rain, snow and a grid-box fractional cloud cover. The cloud 
scheme represents the sources and sinks of cloud and precipitation due 
to the major generation and destruction processes, including cloud formation 
by detrainment from cumulus convection, condensation, ice deposition, evaporation, 
hydrometeor collection, melting and freezing. The scheme is based on \cite{Tiedtke1993} 
but with an enhanced representation of the ice-phase in clouds and precipitation. 
A multi-dimensional implicit solver is used for the numerical solution of 
the cloud and precipitation prognostic equations.  A more detailed description 
of the formulation of the parametrization can be found in \cite{IFSdoc} with 
further discussion in \cite{ForbesandTompkins2011} and \cite{Forbesetal2011}.



\subsection{Governing equations}
The equations for the tendency of the grid-box averaged cloud
liquid, cloud ice, rain and snow water contents are

\begin{equation}
\frac{\partial q_{\text{l}}}{\partial t} =A(q_{\text{l}}) +S_{\text{conv}} 
+S_{\text{strat}}+S_{\text{melt}}^{\text{ice}}-S_{\text{dep}}^{\text{ice}}
-S_{\text{evap}}^{\text{liq}}-S_{\text{auto}}^{\text{rain}}-S_{\text {rime}}^{\text{snow}}
\end{equation}
\begin{equation}
\frac{\partial q_{\text{i}}}{\partial t} =A(q_{\text{i}}) +S_{\text{conv}} 
+S_{\text{strat}}+S_{\text{dep}}^{\text{ice}}-S_{\text{melt}}^{\text{ice}}
-S_{\text{evap}}^{\text{ice}}-S_{\text{auto}}^{\text{snow}}
\end{equation}
\begin{equation}
\frac{\partial q_{\text{r}}}{\partial t} =A(q_{\text{r}}) 
-S_{\text{evap}}^{\text{rain}}+S_{\text{auto}}^{\text{rain}}+S_{\text{melt}^{\text snow}} 
-S_{\text{frz}}^{\text{rain}}  
\end{equation}
\begin{equation}
\frac{\partial q_{\text{s}}}{\partial t} =A(q_{\text{s}}) 
-S_{\text{evap}}^{\text{snow}}+S_{\text{auto}}^{\text{snow}}-S_{\text{melt}}^{\text{snow}} 
+S_{\text{frz}}^{\text{rain}}+S_{\text{rime}}^{\text{snow}}
\end{equation}

and for the cloud fraction,

\begin{equation}
\frac{\partial a}{\partial t} =A(a) +\delta a_{\text{conv}} 
+\delta a_{\text{strat}}-\delta a_{\text{evap}}
\end{equation}

The terms on the right-hand side represent the following processes:
\begin{itemize}
\item    $A(q), A(a)$  -- rate of change of water contents and cloud
area due to transport through the boundaries of the grid volume (advection, 
sedimentation).
\item    $S_{\text{conv}}, \delta a_{\text{conv}}$ -- rate of formation 
of cloud water/ice and cloud area by convective processes.
\item    $S_{\text{strat}}, \delta a_{\text{strat}}$ -- rate of formation 
of cloud water/ice and cloud area by stratiform condensation processes.
\item    $S_{\text{evap}}$ -- rate of evaporation of cloud water/ice, 
rain/snow. 
\item    $S_{\text{auto}}$ -- rate of generation of precipitation from
cloud water/ice (autoconversion).
\item    $S_{\text{melt}}$ -- rate of melting ice/snow.
\item    $S_{\text{rime}}$ -- rate of riming (collection of cloud liquid 
drops).
\item    $S_{\text{frz}} $ -- rate of freezing of rain.
\item    $\delta a_{\text{evap}}$  -- rate of decrease of cloud area due 
to evaporation.
\end{itemize}

The large-scale budget equations for specific humidity $q_v$, 
and dry static energy $s =c_pT +gz$ in the cloud scheme are
\begin{equation}
\frac{\partial q_{\text{v}}}{\partial t} =A(q_{\text{v}}) -S_{\text{strat}} +
S_{\text{evap}}
\end{equation}
and
\begin{equation}
\frac{\partial s}{\partial t} =A(s)+L_{\text{vap}}(S_{\text{strat}} -
S_{\text{evap}})+L_{\text{fus}}(S_{\text{frz}} + S_{\text{rime}} - S_{\text{melt}})
\end{equation}
where $A(q_v)$ and $A(s)$ represent all processes except those
related to clouds, $L_{\text{vap}}$ is the latent heat
of condensation and $L_{\text{fus}}$ is the latent heat
of freezing.

Each of the microphysical source and sink terms is represented by an equation 
or set of equations that vary in complexity, from a simple linear form to more 
non-linear functions involving exponentials and power laws. Some terms are formulated
explicitly and others implicitly and they are combined in a multi-dimensional solver 
to produce the tendencies for the prognostic variables
(cloud liquid, cloud ice, rain, snow and humidity). Cloud fraction is treated
separately as this is a non-conservative variable. The temperature tendency due
to change in phase (vapour, liquid, ice) is calculated after the solver once the
final tendencies are known. 


%-------------------------------------------------------------------------------
\subsection{Integration of the equations}
  
The above equations governing the tendency for each prognostic cloud
variable within the cloud scheme can be written as:
\begin{equation}
\frac{\partial {q}_x}{\partial t} = A_x +
\frac{1}{\rho}\frac{\partial}{\partial z} \left( \rho V_x {q}_x \right)
\end{equation}
where $q_x$ is the specific water content for category $x$ (so $x=1$
represents cloud liquid, $x=2$ for rain, and so on), $A_x$ is the net source
or sink of $q_x$ through microphysical processes, and the last term
represents the sedimentation of $q_x$ with fall speed $V_x$.

The solution to this set of equations uses the upstream approach. Writing the
advection term in mass flux form and collecting all fast processes (relative to
the model timestep) into an implicit term, gives:
\begin{equation}
\frac{q_x^{n+1}-q_x^{n}}{\Delta t} = A_x 
+\sum_{x=1}^{m}B_{xy}q_y^{n+1}
-\sum_{x=1}^{m}B_{yx}q_x^{n+1}
+\frac{ 
\rho_{k-1} V_{x} q_{x,k-1}^{n+1} - \rho V_{x} q_{x}^{n+1} }{ \rho \Delta
Z}
\label{eqn-impl}
\end{equation}

for timestep $n$. The subscript "$k-1$" refers to a term calculated at the model
level above the present level $k$ for which all other terms are being
calculated. The matrix $\widetilde{B}$ (with terms $B_{xx}$, $B_{xy}$, $B_{yx}$)
represents all the implicit microphysical pathways such that $B_{xy}>0$
represents a sink of $q_y$ and a source of $q_x$. Matrix $\widetilde{B}$ is
positive-definite off the diagonal, with zero diagonal terms since $B_{xx}=0$ by
definition.  Some terms, such as the creation of cloud through condensation
resulting from adiabatic motion or diabatic heating, are more suitable for an
explicit framework, and are retained in the explicit term $A$.

For cloud fraction, there are no multi-dimensional dependencies, so the equation
simplifies to
\begin{equation}
\frac{a^{n+1}-a^{n}}{\Delta t} = A + Ba^{n+1}
\end{equation}

However, for the cloud and precipitation variables,  a
matrix approach is required. Due to the cross-terms $q_y^{n+1}$, 
(\ref{eqn-impl}) is rearranged to give a straight forward matrix equation which
can be solved with standard methods. 
The solution method is simplified by assuming the vertical advection terms due
to convective subsidence and sedimentation act only in the downward direction,
allowing the solution to be conducted level by level from the model top down.

The matrix on the left-hand side has the microphysical terms in isolation off the diagonal,
with the sedimentation term on the diagonal, thus the matrix equation for a
3-variable system is
{\footnotesize
\begin{eqnarray}
\left(
\begin{array}{ccc}
1+ \Delta t(\frac{V_1}{\Delta z} + B_{21} + B_{31}) & -\Delta t B_{12}
& -\Delta t B_{13} \\
-\Delta t B_{21} & 1+ \Delta t(\frac{V_2}{\Delta z} + B_{12} + B_{32}) &
-\Delta t B_{23}  \\
-\Delta t B_{31} & -\Delta t B_{32} & 1+ \Delta t(\frac{V_3}{\Delta z} 
+ B_{13} + B_{23}) \\
\end{array}
\right) \cdot
\left(
\begin{array}{c}
q_1^{n+1} \\
q_2^{n+1} \\
q_3^{n+1} \\
\end{array}
\right)
= \nonumber \\
\newline
\left[ 
q_1^{n} + \Delta t \left(A_1 +
\frac{ \rho_{k-1} V_{1} q_{1,k-1}^{n+1}}{\rho \Delta Z} \right) 
,q_2^{n} + \Delta t \left(A_2 +
\frac{ \rho_{k-1} V_{2} q_{2,k-1}^{n+1}}{\rho \Delta Z} \right) 
,q_3^{n} + \Delta t \left(A_3 +
\frac{ \rho_{k-1} V_{3} q_{3,k-1}^{n+1}}{\rho \Delta Z} \right) 
\right].
\end{eqnarray}
}

There are some aspects that require attention.  Firstly, although
implicit terms are unable to reduce a cloud category to zero, the
explicit can, and often will, achieve this.  Thus safety checks are
required to ensure that all end-of-timestep variables remain positive
definite, in addition to ensuring conservation.  Practically, to aid
the conservation requirement, the explicit source and sink terms are
thus also generalised from a vector $\vec A$ to an anti-symmetric
matrix  $\widetilde{A}$,
\begin{eqnarray}
\widetilde{A}=\left(
\begin{array}{ccc}
A_{11} & A_{21} & A_{31} \\
-A_{12} & A_{22} & A_{32} \\
-A_{13} & -A_{23} & A_{33} \\
\end{array}
\right)
\end{eqnarray}

Thus $A_{xy}>0$ represents a source of $q_x$ and a sink of $q_y$, and
the original vector for $A$ can be obtained by summing over the rows.
Although this matrix approach involves a degree of redundancy, it is a
simple method of ensuring conservation properties. The matrix
diagonals $A_{xx}$ contain the 'external' sources of $q_x$ such as the
cloud water detrainment terms from the convection scheme.

In order to simultaneously guarantee conservation and
positive-definite properties, the sum of all sinks for 
a given variable are scaled to avoid negative values. 



\section{Dwarf usage and testing}
In this section we describe how to download and install 
the dwarf along with all its dependencies and we show 
how to run it for a simple test case.



\subsection{Download and installation}
The first step is to download and install the dwarf along 
with all its dependencies. With this purpose, it is possible 
to use the script provided under the ESCAPE software collaboration 
platform:\\
\url{https://software.ecmwf.int/stash/projects/ESCAPE}.

Here you can find a repository called \inlsh{escape}.
You need to download it. You could first create a 
folder named, for instance, ESCAPE, enter into it 
and subsequently download the repository. 
This can be done following the steps below:
%
\begin{lstlisting}[style=BashStyle]
mkdir ESCAPE
cd ESCAPE/
git clone ssh://git@software.ecmwf.int:7999/escape/escape.git
\end{lstlisting}
%
Once downloaded the repository into the \inlsh{ESCAPE} folder 
just created, you should find a new folder called \inlsh{escape}. 
The folder contains a sub-folder called \inlsh{bin} that has the 
python/bash script (called \inlsh{escape}) that needs to be 
run for downloading and installing the dwarf and its dependencies. 
To see the various options provided by the script you can type:
%
\begin{lstlisting}[style=BashStyle]
./escape/bin/escape -h
\end{lstlisting}
%
To download the dwarf and its dependencies you need to run 
the following command:
%
\begin{lstlisting}[style=BashStyle]
./escape/bin/escape checkout dwarf3-cloudscheme --ssh
\end{lstlisting}
% 
Note the option \inlsh{--ssh}. This can be used only internally 
at ECMWF for the moment. If you are an external project partner
you should use the following command instead:
%
\begin{lstlisting}[style=BashStyle]
./escape/bin/escape checkout dwarf3-cloudscheme --user <username>
\end{lstlisting}
% 
and follow the instructions. 
The commands above automatically check out the \inlsh{develop}
version of the dwarf. If you want to download a specific branch 
of this dwarf, you can do so by typing:
%
\begin{lstlisting}[style=BashStyle]
./escape/bin/escape checkout dwarf3-cloudscheme --ssh \
--version <branch-name>
\end{lstlisting}
% 
An analogous approach can be used for the \inlsh{-\,-user} 
version of the command. You should now have a folder called 
\inlsh{dwarf3-cloudscheme}.

In the above command, you can specify several other optional 
parameters. To see all these options and how to use them you 
can type the following command:
%
\begin{lstlisting}[style=BashStyle]
./escape checkout -h
\end{lstlisting}
%

At this stage it is possible to install the dwarf 
and all its dependencies. This can be done in two 
different ways. The first way is to compile and 
install each dependency and the dwarf separately:
%
\begin{lstlisting}[style=BashStyle]
./escape/bin/escape generate_install dwarf3-cloudscheme
\end{lstlisting}
% 
The command above will generate a script 
called \inlsh{install-dwarf3-cloudscheme} 
that can be run by typing:
%
\begin{lstlisting}[style=BashStyle]
./install-dwarf3-cloudscheme
\end{lstlisting}
%
This last step will build and install the dwarf 
along with all its dependencies in the following 
paths:
%
\begin{lstlisting}[style=BashStyle]
dwarf3-cloudscheme/builds/
dwarf3-cloudscheme/install/
\end{lstlisting}
%

The second way is to create a bundle that compile 
and install all the dependencies together:
%
\begin{lstlisting}[style=BashStyle]
./escape/bin/escape generate_bundle dwarf3-cloudscheme
\end{lstlisting}
% 
This command will create an infrastructure to avoid
compiling the single third-party libraries individually
when some modifications are applied locally to one of 
them. To complete the compilation and installation process, 
after having run the above command for the bundle, simply 
follow the instructions on the terminal.

In the commands above that generate the installation 
file, you can specify several other optional parameters. 
To see all these options and how to use them you 
can type the following command:
%
\begin{lstlisting}[style=BashStyle]
./escape generate-install -h
./escape generate-bundle -h
\end{lstlisting}
%

\subsection{Testing}
Not supported yet.

%You should now verify that the dwarf works as expected.
%With this purpose, we created a testing framework that
%allows us to verify that the main features of the dwarf 
%are working correctly.

%To run this verification, you should run the following 
%command:
%%
%\begin{lstlisting}[style=BashStyle]
%ctest -j<number-of-tasks>
%\end{lstlisting}
%%
%from inside the \inlsh{builds/dwarf3-cloudscheme}
%folder.
%%
%\begin{warningbox}
%We strongly advise you to verify via ctest that 
%the main functionalities of the dwarf are working 
%properly any time you apply modifications to the 
%code. Updates that do not pass the tests cannot 
%be merged. 
%In addition, if you add a new feature to the dwarf,
%this should be supported by a test if the existing
%testing framework is not already able to verify its
%functionality.
%\end{warningbox}
%%
%For instructions on how to run the executables 
%see the next section.


\section{Run the dwarf}
To run a simple test case you need to link the input file, called \inlsh{cloudsc.bin}, 
required by this dwarf to the directory from where you intend to run the excutable.
This is located in the \inlsh{config-files} subfolder inside the \inlsh{dwarf3-cloudscheme} 
directory and the link can be created as follows
%
\begin{lstlisting}[style=BashStyle]
ln -s dwarf3-cloudscheme/sources/data/cloudsc.bin .
\end{lstlisting}
%
The input data cloudsc.bin is a Fortran unformatted stream binary (no record delimiters). 
It contains data for just 100 grid point columns and will be inflated to full spectre 
of NGPTOT, where NGPTOT is the number of grid point columns.
To run the a simple test you can now type:
%
\begin{lstlisting}[style=BashStyle]
dwarf3-cloudscheme/builds/bin/dwarf3-cloudscheme OMP NGPTOT NPROMA-list
\end{lstlisting}
%
where OMP is the number of threads for OMP-parallel regions, NGPTOT is the already 
mentioned number of grid point columns, NPROMA-list is a list of NPROMAs to use.
For a simple test-case you can try the following parameter-combination:
%
\begin{lstlisting}[style=BashStyle]
dwarf3-cloudscheme/builds/bin/dwarf3-cloudsc 4 160000 2
\end{lstlisting}
%
Note that if you installed the dwarf through the bundle option, the executable 
can be run with the following command:
%
\begin{lstlisting}[style=BashStyle]
dwarf3-cloudscheme/builds/bundle/bin/dwarf3-cloudsc 4 160000 2
\end{lstlisting}
%

 


\section{Integration}
The cloud scheme is part of the physical processes required by any NWP model.
Its inteegration is therefore naturally guaranteed independently from the 
dynamical core being used.


\backmatter

%%% BIBLIOGRAPHY
%%% -------------------------------------------------------------

\bibliographystyle{plain}
\bibliography{refs}


\end{document}
